% Options for packages loaded elsewhere
\PassOptionsToPackage{unicode}{hyperref}
\PassOptionsToPackage{hyphens}{url}
\PassOptionsToPackage{dvipsnames,svgnames,x11names}{xcolor}
%
\documentclass[
  letterpaper,
  DIV=11,
  numbers=noendperiod]{scrartcl}

\usepackage{amsmath,amssymb}
\usepackage{lmodern}
\usepackage{iftex}
\ifPDFTeX
  \usepackage[T1]{fontenc}
  \usepackage[utf8]{inputenc}
  \usepackage{textcomp} % provide euro and other symbols
\else % if luatex or xetex
  \usepackage{unicode-math}
  \defaultfontfeatures{Scale=MatchLowercase}
  \defaultfontfeatures[\rmfamily]{Ligatures=TeX,Scale=1}
\fi
% Use upquote if available, for straight quotes in verbatim environments
\IfFileExists{upquote.sty}{\usepackage{upquote}}{}
\IfFileExists{microtype.sty}{% use microtype if available
  \usepackage[]{microtype}
  \UseMicrotypeSet[protrusion]{basicmath} % disable protrusion for tt fonts
}{}
\makeatletter
\@ifundefined{KOMAClassName}{% if non-KOMA class
  \IfFileExists{parskip.sty}{%
    \usepackage{parskip}
  }{% else
    \setlength{\parindent}{0pt}
    \setlength{\parskip}{6pt plus 2pt minus 1pt}}
}{% if KOMA class
  \KOMAoptions{parskip=half}}
\makeatother
\usepackage{xcolor}
\setlength{\emergencystretch}{3em} % prevent overfull lines
\setcounter{secnumdepth}{5}
% Make \paragraph and \subparagraph free-standing
\ifx\paragraph\undefined\else
  \let\oldparagraph\paragraph
  \renewcommand{\paragraph}[1]{\oldparagraph{#1}\mbox{}}
\fi
\ifx\subparagraph\undefined\else
  \let\oldsubparagraph\subparagraph
  \renewcommand{\subparagraph}[1]{\oldsubparagraph{#1}\mbox{}}
\fi

\usepackage{color}
\usepackage{fancyvrb}
\newcommand{\VerbBar}{|}
\newcommand{\VERB}{\Verb[commandchars=\\\{\}]}
\DefineVerbatimEnvironment{Highlighting}{Verbatim}{commandchars=\\\{\}}
% Add ',fontsize=\small' for more characters per line
\usepackage{framed}
\definecolor{shadecolor}{RGB}{241,243,245}
\newenvironment{Shaded}{\begin{snugshade}}{\end{snugshade}}
\newcommand{\AlertTok}[1]{\textcolor[rgb]{0.68,0.00,0.00}{#1}}
\newcommand{\AnnotationTok}[1]{\textcolor[rgb]{0.37,0.37,0.37}{#1}}
\newcommand{\AttributeTok}[1]{\textcolor[rgb]{0.40,0.45,0.13}{#1}}
\newcommand{\BaseNTok}[1]{\textcolor[rgb]{0.68,0.00,0.00}{#1}}
\newcommand{\BuiltInTok}[1]{\textcolor[rgb]{0.00,0.23,0.31}{#1}}
\newcommand{\CharTok}[1]{\textcolor[rgb]{0.13,0.47,0.30}{#1}}
\newcommand{\CommentTok}[1]{\textcolor[rgb]{0.37,0.37,0.37}{#1}}
\newcommand{\CommentVarTok}[1]{\textcolor[rgb]{0.37,0.37,0.37}{\textit{#1}}}
\newcommand{\ConstantTok}[1]{\textcolor[rgb]{0.56,0.35,0.01}{#1}}
\newcommand{\ControlFlowTok}[1]{\textcolor[rgb]{0.00,0.23,0.31}{#1}}
\newcommand{\DataTypeTok}[1]{\textcolor[rgb]{0.68,0.00,0.00}{#1}}
\newcommand{\DecValTok}[1]{\textcolor[rgb]{0.68,0.00,0.00}{#1}}
\newcommand{\DocumentationTok}[1]{\textcolor[rgb]{0.37,0.37,0.37}{\textit{#1}}}
\newcommand{\ErrorTok}[1]{\textcolor[rgb]{0.68,0.00,0.00}{#1}}
\newcommand{\ExtensionTok}[1]{\textcolor[rgb]{0.00,0.23,0.31}{#1}}
\newcommand{\FloatTok}[1]{\textcolor[rgb]{0.68,0.00,0.00}{#1}}
\newcommand{\FunctionTok}[1]{\textcolor[rgb]{0.28,0.35,0.67}{#1}}
\newcommand{\ImportTok}[1]{\textcolor[rgb]{0.00,0.46,0.62}{#1}}
\newcommand{\InformationTok}[1]{\textcolor[rgb]{0.37,0.37,0.37}{#1}}
\newcommand{\KeywordTok}[1]{\textcolor[rgb]{0.00,0.23,0.31}{#1}}
\newcommand{\NormalTok}[1]{\textcolor[rgb]{0.00,0.23,0.31}{#1}}
\newcommand{\OperatorTok}[1]{\textcolor[rgb]{0.37,0.37,0.37}{#1}}
\newcommand{\OtherTok}[1]{\textcolor[rgb]{0.00,0.23,0.31}{#1}}
\newcommand{\PreprocessorTok}[1]{\textcolor[rgb]{0.68,0.00,0.00}{#1}}
\newcommand{\RegionMarkerTok}[1]{\textcolor[rgb]{0.00,0.23,0.31}{#1}}
\newcommand{\SpecialCharTok}[1]{\textcolor[rgb]{0.37,0.37,0.37}{#1}}
\newcommand{\SpecialStringTok}[1]{\textcolor[rgb]{0.13,0.47,0.30}{#1}}
\newcommand{\StringTok}[1]{\textcolor[rgb]{0.13,0.47,0.30}{#1}}
\newcommand{\VariableTok}[1]{\textcolor[rgb]{0.07,0.07,0.07}{#1}}
\newcommand{\VerbatimStringTok}[1]{\textcolor[rgb]{0.13,0.47,0.30}{#1}}
\newcommand{\WarningTok}[1]{\textcolor[rgb]{0.37,0.37,0.37}{\textit{#1}}}

\providecommand{\tightlist}{%
  \setlength{\itemsep}{0pt}\setlength{\parskip}{0pt}}\usepackage{longtable,booktabs,array}
\usepackage{calc} % for calculating minipage widths
% Correct order of tables after \paragraph or \subparagraph
\usepackage{etoolbox}
\makeatletter
\patchcmd\longtable{\par}{\if@noskipsec\mbox{}\fi\par}{}{}
\makeatother
% Allow footnotes in longtable head/foot
\IfFileExists{footnotehyper.sty}{\usepackage{footnotehyper}}{\usepackage{footnote}}
\makesavenoteenv{longtable}
\usepackage{graphicx}
\makeatletter
\def\maxwidth{\ifdim\Gin@nat@width>\linewidth\linewidth\else\Gin@nat@width\fi}
\def\maxheight{\ifdim\Gin@nat@height>\textheight\textheight\else\Gin@nat@height\fi}
\makeatother
% Scale images if necessary, so that they will not overflow the page
% margins by default, and it is still possible to overwrite the defaults
% using explicit options in \includegraphics[width, height, ...]{}
\setkeys{Gin}{width=\maxwidth,height=\maxheight,keepaspectratio}
% Set default figure placement to htbp
\makeatletter
\def\fps@figure{htbp}
\makeatother
\newlength{\cslhangindent}
\setlength{\cslhangindent}{1.5em}
\newlength{\csllabelwidth}
\setlength{\csllabelwidth}{3em}
\newlength{\cslentryspacingunit} % times entry-spacing
\setlength{\cslentryspacingunit}{\parskip}
\newenvironment{CSLReferences}[2] % #1 hanging-ident, #2 entry spacing
 {% don't indent paragraphs
  \setlength{\parindent}{0pt}
  % turn on hanging indent if param 1 is 1
  \ifodd #1
  \let\oldpar\par
  \def\par{\hangindent=\cslhangindent\oldpar}
  \fi
  % set entry spacing
  \setlength{\parskip}{#2\cslentryspacingunit}
 }%
 {}
\usepackage{calc}
\newcommand{\CSLBlock}[1]{#1\hfill\break}
\newcommand{\CSLLeftMargin}[1]{\parbox[t]{\csllabelwidth}{#1}}
\newcommand{\CSLRightInline}[1]{\parbox[t]{\linewidth - \csllabelwidth}{#1}\break}
\newcommand{\CSLIndent}[1]{\hspace{\cslhangindent}#1}

\KOMAoption{captions}{tableheading}
\makeatletter
\@ifpackageloaded{tcolorbox}{}{\usepackage[many]{tcolorbox}}
\@ifpackageloaded{fontawesome5}{}{\usepackage{fontawesome5}}
\definecolor{quarto-callout-color}{HTML}{909090}
\definecolor{quarto-callout-note-color}{HTML}{0758E5}
\definecolor{quarto-callout-important-color}{HTML}{CC1914}
\definecolor{quarto-callout-warning-color}{HTML}{EB9113}
\definecolor{quarto-callout-tip-color}{HTML}{00A047}
\definecolor{quarto-callout-caution-color}{HTML}{FC5300}
\definecolor{quarto-callout-color-frame}{HTML}{acacac}
\definecolor{quarto-callout-note-color-frame}{HTML}{4582ec}
\definecolor{quarto-callout-important-color-frame}{HTML}{d9534f}
\definecolor{quarto-callout-warning-color-frame}{HTML}{f0ad4e}
\definecolor{quarto-callout-tip-color-frame}{HTML}{02b875}
\definecolor{quarto-callout-caution-color-frame}{HTML}{fd7e14}
\makeatother
\makeatletter
\makeatother
\makeatletter
\makeatother
\makeatletter
\@ifpackageloaded{caption}{}{\usepackage{caption}}
\AtBeginDocument{%
\ifdefined\contentsname
  \renewcommand*\contentsname{Table of contents}
\else
  \newcommand\contentsname{Table of contents}
\fi
\ifdefined\listfigurename
  \renewcommand*\listfigurename{List of Figures}
\else
  \newcommand\listfigurename{List of Figures}
\fi
\ifdefined\listtablename
  \renewcommand*\listtablename{List of Tables}
\else
  \newcommand\listtablename{List of Tables}
\fi
\ifdefined\figurename
  \renewcommand*\figurename{Figure}
\else
  \newcommand\figurename{Figure}
\fi
\ifdefined\tablename
  \renewcommand*\tablename{Table}
\else
  \newcommand\tablename{Table}
\fi
}
\@ifpackageloaded{float}{}{\usepackage{float}}
\floatstyle{ruled}
\@ifundefined{c@chapter}{\newfloat{codelisting}{h}{lop}}{\newfloat{codelisting}{h}{lop}[chapter]}
\floatname{codelisting}{Listing}
\newcommand*\listoflistings{\listof{codelisting}{List of Listings}}
\makeatother
\makeatletter
\@ifpackageloaded{caption}{}{\usepackage{caption}}
\@ifpackageloaded{subcaption}{}{\usepackage{subcaption}}
\makeatother
\makeatletter
\@ifpackageloaded{tcolorbox}{}{\usepackage[many]{tcolorbox}}
\makeatother
\makeatletter
\@ifundefined{shadecolor}{\definecolor{shadecolor}{rgb}{.97, .97, .97}}
\makeatother
\makeatletter
\makeatother
\ifLuaTeX
  \usepackage{selnolig}  % disable illegal ligatures
\fi
\IfFileExists{bookmark.sty}{\usepackage{bookmark}}{\usepackage{hyperref}}
\IfFileExists{xurl.sty}{\usepackage{xurl}}{} % add URL line breaks if available
\urlstyle{same} % disable monospaced font for URLs
\hypersetup{
  pdftitle={Data Science for Public Health - Day 1 - Import data},
  colorlinks=true,
  linkcolor={blue},
  filecolor={Maroon},
  citecolor={Blue},
  urlcolor={Blue},
  pdfcreator={LaTeX via pandoc}}

\title{Data Science for Public Health - Day 1 - Import data}
\author{}
\date{2022-09-26}

\begin{document}
\maketitle
\ifdefined\Shaded\renewenvironment{Shaded}{\begin{tcolorbox}[borderline west={3pt}{0pt}{shadecolor}, enhanced, sharp corners, frame hidden, interior hidden, boxrule=0pt, breakable]}{\end{tcolorbox}}\fi

\renewcommand*\contentsname{Table of contents}
{
\hypersetup{linkcolor=}
\setcounter{tocdepth}{3}
\tableofcontents
}
\hypertarget{introduction}{%
\section{Introduction}\label{introduction}}

Most of the time you will want to generate \emph{Quarto} documents using
your own data.

One of the first thing you will usually do when starting a new
\emph{Quarto} document is to import your data.

If you want to further process / analyse your data, you will have to
store your data in a two-dimensional array-like structure (\textbf{data
frame}) that contains rows and columns. You can store multiple data sets
in memory (which will be stored in different data frames) and work on
all of them in parallel.

\begin{tcolorbox}[enhanced jigsaw, arc=.35mm, coltitle=black, toprule=.15mm, bottomtitle=1mm, colbacktitle=quarto-callout-warning-color!10!white, bottomrule=.15mm, titlerule=0mm, rightrule=.15mm, colframe=quarto-callout-warning-color-frame, title=\textcolor{quarto-callout-warning-color}{\faExclamationTriangle}\hspace{0.5em}{Warning}, breakable, opacityback=0, opacitybacktitle=0.6, toptitle=1mm, leftrule=.75mm, colback=white, left=2mm]

For (advanced) Python users only: you need to import the R package
\texttt{reticulate} if you want to use the Knitr engine and manipulate
Python objects within R code chunks.

\begin{Shaded}
\begin{Highlighting}[]
\CommentTok{\# Write your code here}
\end{Highlighting}
\end{Shaded}

\end{tcolorbox}

\hypertarget{import-data-from-files}{%
\section{Import data from files}\label{import-data-from-files}}

There is a dedicated importing function in R and Python for almost every
data format. In this session we show you how to read Stata
(\texttt{.dta}), Excel (\texttt{.xlsx}) and comma-separated values (CSV,
\texttt{.csv}) formats.

Only one argument is required within these function. We need to know the
PATH where the file is stored.

Indicate functions that accepts URL as well (it is the case for Python
functions).

\hypertarget{import-excel-data}{%
\subsection{Import Excel data}\label{import-excel-data}}

Read the Excel data set \textbf{dataset1.xlsx} and store it into a data
frame called \textbf{df1}.

\begin{tcolorbox}[enhanced jigsaw, arc=.35mm, coltitle=black, toprule=.15mm, bottomtitle=1mm, colbacktitle=quarto-callout-tip-color!10!white, bottomrule=.15mm, titlerule=0mm, rightrule=.15mm, colframe=quarto-callout-tip-color-frame, title=\textcolor{quarto-callout-tip-color}{\faLightbulb}\hspace{0.5em}{Tip}, breakable, opacityback=0, opacitybacktitle=0.6, toptitle=1mm, leftrule=.75mm, colback=white, left=2mm]

\begin{itemize}
\tightlist
\item
  R: use the
  \href{https://readxl.tidyverse.org/reference/read_excel.html}{read\_excel}
  function from the \texttt{readxl} package.
\item
  Python: use the
  \href{https://pandas.pydata.org/docs/reference/api/pandas.read_excel.html}{read\_xls}
  function from the \texttt{pandas} package.
\end{itemize}

\end{tcolorbox}

\begin{Shaded}
\begin{Highlighting}[]
\CommentTok{\# Write your code here}
\end{Highlighting}
\end{Shaded}

\hypertarget{import-csv-data}{%
\subsection{Import CSV data}\label{import-csv-data}}

Read the CSV data set \textbf{dataset1.csv} and store it into a data
frame called \textbf{df2}.

\begin{tcolorbox}[enhanced jigsaw, arc=.35mm, coltitle=black, toprule=.15mm, bottomtitle=1mm, colbacktitle=quarto-callout-tip-color!10!white, bottomrule=.15mm, titlerule=0mm, rightrule=.15mm, colframe=quarto-callout-tip-color-frame, title=\textcolor{quarto-callout-tip-color}{\faLightbulb}\hspace{0.5em}{Tip}, breakable, opacityback=0, opacitybacktitle=0.6, toptitle=1mm, leftrule=.75mm, colback=white, left=2mm]

\begin{itemize}
\tightlist
\item
  R: use the
  \href{https://www.rdocumentation.org/packages/utils/versions/3.6.2/topics/read.table}{read.csv}
  function from the \texttt{haven} package.
\item
  Python: use the
  \href{https://pandas.pydata.org/docs/reference/api/pandas.read_csv.html}{read\_csv}
  function from the \texttt{pandas} package.
\end{itemize}

\end{tcolorbox}

\begin{Shaded}
\begin{Highlighting}[]
\CommentTok{\# Write your code here}
\end{Highlighting}
\end{Shaded}

\hypertarget{import-stata-data}{%
\subsection{Import Stata data}\label{import-stata-data}}

Read the Stata data set \textbf{dataset1.dta} and store it into a data
frame called \textbf{df3}.

\begin{tcolorbox}[enhanced jigsaw, arc=.35mm, coltitle=black, toprule=.15mm, bottomtitle=1mm, colbacktitle=quarto-callout-tip-color!10!white, bottomrule=.15mm, titlerule=0mm, rightrule=.15mm, colframe=quarto-callout-tip-color-frame, title=\textcolor{quarto-callout-tip-color}{\faLightbulb}\hspace{0.5em}{Tip}, breakable, opacityback=0, opacitybacktitle=0.6, toptitle=1mm, leftrule=.75mm, colback=white, left=2mm]

\begin{itemize}
\tightlist
\item
  R: use the
  \href{https://haven.tidyverse.org/reference/read_dta.html}{read\_dta}
  function from the \texttt{haven} package. This package supports SAS,
  STATA and SPSS software.
\item
  Python: use the
  \href{https://pandas.pydata.org/docs/reference/api/pandas.read_stata.html}{read\_stata}
  function from the \texttt{pandas} package.
\end{itemize}

\end{tcolorbox}

\begin{Shaded}
\begin{Highlighting}[]
\CommentTok{\# Write your code here}
\end{Highlighting}
\end{Shaded}

\hypertarget{additional-question}{%
\subsection{Additional question}\label{additional-question}}

Can you indicate what variable has been modified in \textbf{dataset1}
between df1 and df2?

\begin{tcolorbox}[enhanced jigsaw, arc=.35mm, coltitle=black, toprule=.15mm, bottomtitle=1mm, colbacktitle=quarto-callout-tip-color!10!white, bottomrule=.15mm, titlerule=0mm, rightrule=.15mm, colframe=quarto-callout-tip-color-frame, title=\textcolor{quarto-callout-tip-color}{\faLightbulb}\hspace{0.5em}{Tip}, breakable, opacityback=0, opacitybacktitle=0.6, toptitle=1mm, leftrule=.75mm, colback=white, left=2mm]
Use the R function \textbf{comparedf}
\end{tcolorbox}

\begin{Shaded}
\begin{Highlighting}[]
\CommentTok{\# Write your code here}
\end{Highlighting}
\end{Shaded}

\begin{verbatim}
Compare Object

Function Call: 
arsenal::comparedf(x = df1, y = df2)

Shared: 0 non-by variables and 32 observations.
Not shared: 16 variables and 118 observations.

Differences found in 0/0 variables compared.
0 variables compared have non-identical attributes.
\end{verbatim}

Can you indicate what variable has been modified in \textbf{dataset1}
between df1 and df3?

\begin{Shaded}
\begin{Highlighting}[]
\CommentTok{\# Write your code here}
\end{Highlighting}
\end{Shaded}

\begin{verbatim}
Compare Object

Function Call: 
arsenal::comparedf(x = df1, y = py$df3)

Shared: 0 non-by variables and 32 observations.
Not shared: 16 variables and 118 observations.

Differences found in 0/0 variables compared.
0 variables compared have non-identical attributes.
\end{verbatim}

\hypertarget{import-data-from-odk-central}{%
\section{Import data from ODK
Central}\label{import-data-from-odk-central}}

While data digitally captured using ODK Collect or Enketo and stored in
ODK Central can be retrieved in bulk through the web interface, ODK
Central's API provides direct access to its data (and functionality). It
is a more efficient way to ensure that data are always up-to-date.

We will see how to retrieve data through the OData API.

\begin{tcolorbox}[enhanced jigsaw, arc=.35mm, coltitle=black, toprule=.15mm, bottomtitle=1mm, colbacktitle=quarto-callout-important-color!10!white, bottomrule=.15mm, titlerule=0mm, rightrule=.15mm, colframe=quarto-callout-important-color-frame, title=\textcolor{quarto-callout-important-color}{\faExclamation}\hspace{0.5em}{Important}, breakable, opacityback=0, opacitybacktitle=0.6, toptitle=1mm, leftrule=.75mm, colback=white, left=2mm]
For advanced users: encrypted data can only be retrieved through the
RESTful API.
\end{tcolorbox}

\textbf{Instructions:}

\begin{enumerate}
\def\labelenumi{\arabic{enumi}.}
\tightlist
\item
  Connect to the ODK Central server data set (give access to a dummy ODK
  Central project for demo purpose).
\item
  Retrieve data from form.
\item
  Store the data set into a data frame called df4
\end{enumerate}

\begin{tcolorbox}[enhanced jigsaw, arc=.35mm, coltitle=black, toprule=.15mm, bottomtitle=1mm, colbacktitle=quarto-callout-tip-color!10!white, bottomrule=.15mm, titlerule=0mm, rightrule=.15mm, colframe=quarto-callout-tip-color-frame, title=\textcolor{quarto-callout-tip-color}{\faLightbulb}\hspace{0.5em}{Tip}, breakable, opacityback=0, opacitybacktitle=0.6, toptitle=1mm, leftrule=.75mm, colback=white, left=2mm]

\begin{itemize}
\tightlist
\item
  R: use the
  \href{https://docs.ropensci.org/ruODK/reference/ru_setup.html}{ru\_setup}
  and the
  \href{https://docs.ropensci.org/ruODK/reference/odata_submission_get.html}{odata\_submission\_get}
  functions from the \texttt{ruODK} package (Mayer 2020).
\item
  Python: use functions from the \texttt{pyODK} package.
\end{itemize}

\end{tcolorbox}

\begin{Shaded}
\begin{Highlighting}[]
\CommentTok{\# Write your code here}
\end{Highlighting}
\end{Shaded}

\hypertarget{references}{%
\section*{References}\label{references}}
\addcontentsline{toc}{section}{References}

\hypertarget{refs}{}
\begin{CSLReferences}{1}{0}
\leavevmode\vadjust pre{\hypertarget{ref-mayer20}{}}%
Mayer, Florian W. 2020. {``ruODK: Client for the ODK Central API.''}
\url{https://github.com/ropensci/ruODK}.

\end{CSLReferences}



\end{document}
